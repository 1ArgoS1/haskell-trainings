%%%%%%%%%%%%%%%%%%%%%%%%%%%%%%%%%%%%%%%%%%%%%%%%%%%%%%%%%%%%%%%%%%%%%%%%%%%%
%% Header

\documentclass[17pt]{beamer}

\usepackage[english]{babel}
\usepackage{amssymb}
\usepackage{beamerthemesplit}
\usepackage{calc}
\usepackage{colortbl}
\usepackage{fdsymbol}
\usepackage{floatflt}
\usepackage[no-math]{fontspec}
\usepackage{graphicx}
\usepackage{listings}
\usepackage{soul}
\usepackage{xunicode}

% When working on specific pages it is faster to generate just those pages.
% \usepackage[163-180]{pagesel}

\title{Haskell 102}
\author{ibobyr@, nicuveo@}
%\institute{Google}
\date{\small\today}



%%%%%%%%%%%%%%%%%%%%%%%%%%%%%%%%%%%%%%%%%%%%%%%%%%%%%%%%%%%%%%%%%%%%%%%%%%%%
%% Customization

%%
%% theme.tex
%% Made by nicuveo <antoine.jp.leblanc@gmail.com>
%%


%%%%%%%%%%%%%%%%%%%%%%%%%%%%%%%%%%%%%%%%%%%%%%%%%%%%%%%%%%%%%%%%%%%%%%%%%%%%
%% Base theme

\mode<presentation>{}
\usecolortheme{whale}
%\usepackage{fontspec}
\usepackage{etoolbox}
\setsansfont[Ligatures=TeX, % recommended
             UprightFont={*-Regular},
             ItalicFont={*-Regular},
             BoldFont={*-Bold},
             BoldItalicFont={*-Bold}]
             {Yanone Kaffeesatz}

%%%%%%%%%%%%%%%%%%%%%%%%%%%%%%%%%%%%%%%%%%%%%%%%%%%%%%%%%%%%%%%%%%%%%%%%%%%%
%% Colors

% define

\definecolor{google-r}{RGB}{234,  67,  53}
\definecolor{google-g}{RGB}{ 52, 168,  83}
\definecolor{google-b}{RGB}{ 66, 133, 244}
\definecolor{google-y}{RGB}{251, 188,   5}

\definecolor{my-color-1}{RGB}{ 66, 133, 244} % primary
\definecolor{my-color-2}{RGB}{234,  67,  53} % lighter
\definecolor{my-color-3}{RGB}{  0,   0,   0} % unknown
\definecolor{my-color-4}{RGB}{  0,   0,   0} % darker

\definecolor{tab-1}{rgb}{0.04,0.34,0.58}
\definecolor{tab-2}{rgb}{0.36,0.56,0.72}
\definecolor{tab-3}{rgb}{0.68,0.78,0.86}


% set

\setbeamercolor{structure}{fg=my-color-1}
\setbeamercolor{alerted text}{fg=my-color-2}

\setbeamercolor{palette primary}   {fg=white, bg=my-color-1}
\setbeamercolor{palette secondary} {fg=black, bg=my-color-2}
\setbeamercolor{palette tertiary}  {fg=white, bg=my-color-3}
\setbeamercolor{palette quaternary}{fg=white, bg=my-color-4}

\setbeamercolor{page number in head/foot} {fg=black, bg=white}
\setbeamercolor{icon in head/foot} {fg=black, bg=white}

\setbeamercolor*{separation line}{}
\setbeamercolor*{fine separation line}{}



%%%%%%%%%%%%%%%%%%%%%%%%%%%%%%%%%%%%%%%%%%%%%%%%%%%%%%%%%%%%%%%%%%%%%%%%%%%%
%% Templates

\newtoggle{showpagenumber}

\setbeamertemplate{headline}[default]
\defbeamertemplate{footline}{mine}[2]
{
  \leavevmode%
  \hbox
  {%
    \hskip .020\paperwidth

    \begin{beamercolorbox}[wd=.160\paperwidth,ht=3ex,dp=3ex,center]{icon in head/foot}
      \pgfimage[mask=#2,interpolate=true,height=14pt]{#1}
    \end{beamercolorbox}%

    \hskip .720\paperwidth

    \begin{beamercolorbox}[wd=.100\paperwidth,ht=3ex,dp=3ex,center]{page number in head/foot}
      \usebeamerfont{page number in head/foot}
      \iftoggle{showpagenumber}{
        \insertframenumber{} / \inserttotalframenumber{}
      }{}
      \end{beamercolorbox}%
  }%
  \vskip0pt%
}


\newcounter{SectionColorCounter}
\AtBeginSection[]
{
  \ifnum\value{SectionColorCounter}=0
    \setbeamercolor{palette primary}{fg=white, bg=google-g}
    \setbeamercolor{frametitle}{fg=white, bg=google-g}
    \setbeamercolor{structure}{fg=google-g}
    \setbeamercolor{alerted text}{fg=google-g}
  \fi
  \ifnum\value{SectionColorCounter}=1
    \setbeamercolor{palette primary}{fg=white, bg=google-b}
    \setbeamercolor{frametitle}{fg=white, bg=google-b}
    \setbeamercolor{structure}{fg=google-b}
    \setbeamercolor{alerted text}{fg=google-b}
  \fi
  \ifnum\value{SectionColorCounter}=2
    \setbeamercolor{palette primary}{fg=white, bg=google-r}
    \setbeamercolor{frametitle}{fg=white, bg=google-r}
    \setbeamercolor{structure}{fg=google-r}
    \setbeamercolor{alerted text}{fg=google-r}
  \fi
  \ifnum\value{SectionColorCounter}=3
    \setbeamercolor{palette primary}{fg=white, bg=google-y}
    \setbeamercolor{frametitle}{fg=white, bg=google-y}
    \setbeamercolor{structure}{fg=google-y}
    \setbeamercolor{alerted text}{fg=google-y}
  \fi

  \stepcounter{SectionColorCounter}
  \ifnum\value{SectionColorCounter}=4
    \setcounter{SectionColorCounter}{0}
  \fi
}

% \AtBeginSection[]
% {
%    \begin{frame}
%        \frametitle{Outline}
%        \tableofcontents[currentsection,hideothersubsections]
%    \end{frame}
% }

\newcommand{\setfooterlogo}[2]
{
  \setbeamertemplate{footline}[mine]{#1}{#2}
}

%\pgfdeclaremask{masklogo}{img/logo}
\newcommand\defaultlogo{\setfooterlogo{img/google}{}}
\defaultlogo

\setbeamersize{text margin left=10pt,text margin right=10pt}



%%%%%%%%%%%%%%%%%%%%%%%%%%%%%%%%%%%%%%%%%%%%%%%%%%%%%%%%%%%%%%%%%%%%%%%%%%%%
%% Code environnement

\definecolor{hsk-comment}   {gray}{0.5}
\colorlet{hsk-built-ins}    {google-g}
\definecolor{hsk-types}        {rgb}{0, 0.53, 0.69}
\definecolor{hsk-typevar}      {rgb}{0.82, 0.41, 0.12}
% \definecolor{hsk-typevar}      {rgb}{0.65, 0.32, 0.18}
% \definecolor{hsk-typevar}      {rgb}{0.0, 0.0, 0.8}
\definecolor{hsk-constructors} {rgb}{0.85, 0.65, 0.13}
\definecolor{hsk-operators}    {rgb}{0.84, 0.0, 0.53}
\definecolor{hsk-keywords}     {rgb}{0.0, 0.37, 0.69}
\definecolor{hsk-strings}      {rgb}{0.37, 0.53, 0.0}

\lstdefinelanguage{ColorHaskell} {
        basicstyle=\ttfamily\footnotesize,
        sensitive=true,
        morecomment=[l][\color{hsk-comment}\ttfamily\footnotesize]{--},
        morecomment=[s][\color{hsk-comment}\ttfamily\footnotesize]{\{-}{-\}},
        morestring=[b]",
        stringstyle=\color{hsk-strings},
        showstringspaces=false,
        numberstyle=none,
        showspaces=false,
        breaklines=true,
        showtabs=false,
        emph=
        {[1]
                FilePath,IOError,Bool,Char,Double,Either,Float,IO,Integer,Int,Maybe,Ordering,Rational,Ratio,ReadS,ShowS,String, Word8,InPacket
        },
        emphstyle={[1]\color{hsk-types}},
        emph=
        {[2]
                case,class,data,deriving,do,else,if,import,in,infixl,infixr,instance,let,
                module,of,primitive,then,type,where
        },
        emphstyle={[2]\color{hsk-keywords}\textbf},
        emph=
        {[3]
                quot,rem,div,mod,elem,notElem,seq
        },
        emphstyle={[3]\color{hsk-operators}\textbf},
        emph=
        {[4]
                EQ,GT,Just,LT,Left,Nothing,Right
        },
        emphstyle={[4]\color{hsk-constructors}\textbf}
}

\lstnewenvironment{code}
    {\lstset{language=ColorHaskell,basicstyle=\footnotesize\ttfamily}%
      \csname lst@SetFirstLabel\endcsname}
    {\csname lst@SaveFirstLabel\endcsname}
    \lstset{
      basicstyle=\footnotesize\ttfamily,
      flexiblecolumns=false,
      basewidth={0.5em,0.45em},
      escapechar=;,
      literate={`\\}{{$\lambda$}}1
               {`\\\\}{{\char`\\\char`\\}}1
               {`->}{{$\rightarrow$}}2
               {`<-}{{$\leftarrow$}}2
               {`=>}{{$\Rightarrow$}}2
               {`>>}{{>>}}2
               {`>>=}{{>>=}}3
               {`>=>}{{>=>}}3
               {`|}{{$\mid$}}1
    }

\lstdefinelanguage{ColorHaskellWithTypeVars} {
        basicstyle=\ttfamily\footnotesize,
        sensitive=true,
        morecomment=[l][\color{hsk-comment}\ttfamily\footnotesize]{--},
        morecomment=[s][\color{hsk-comment}\ttfamily\footnotesize]{\{-}{-\}},
        morestring=[b]",
        stringstyle=\color{hsk-strings},
        showstringspaces=false,
        numberstyle=none,
        showspaces=false,
        breaklines=true,
        showtabs=false,
        emph=
        {[1]
                FilePath,IOError,Bool,Char,Double,Either,Float,IO,Integer,Int,Maybe,Ordering,Rational,Ratio,ReadS,ShowS,String, Word8,InPacket
        },
        emphstyle={[1]\color{hsk-types}},
        emph=
        {[2]
                case,class,data,deriving,do,else,if,import,in,infixl,infixr,instance,let,
                module,of,primitive,then,type,where
        },
        emphstyle={[2]\color{hsk-keywords}\textbf},
        emph=
        {[3]
                quot,rem,div,mod,elem,notElem,seq
        },
        emphstyle={[3]\color{hsk-operators}\textbf},
        emph=
        {[4]
                EQ,GT,Just,LT,Left,Nothing,Right
        },
        emphstyle={[4]\color{hsk-constructors}\textbf},
        emph=
        {[5]
                a,b,c,m,t
        },
        emphstyle={[5]\color{hsk-typevar}\textbf}
}

\lstnewenvironment{codewithtvars}
    {\lstset{language=ColorHaskellWithTypeVars,basicstyle=\footnotesize\ttfamily}%
      \csname lst@SetFirstLabel\endcsname}
    {\csname lst@SaveFirstLabel\endcsname}
    \lstset{
      basicstyle=\footnotesize\ttfamily,
      flexiblecolumns=false,
      basewidth={0.5em,0.45em},
      escapechar=;,
      literate={`\\}{{$\lambda$}}1
               {`\\\\}{{\char`\\\char`\\}}1
               {`->}{{$\rightarrow$}}2
               {`<-}{{$\leftarrow$}}2
               {`=>}{{$\Rightarrow$}}2
               {`>>}{{>>}}2
               {`>>=}{{>>=}}3
               {`>=>}{{>=>}}3
               {`|}{{$\mid$}}1
    }

\def\inlinecode{\lstinline[language=ColorHaskell,
      basicstyle=\footnotesize\ttfamily,
      flexiblecolumns=false,
      basewidth={0.5em,0.45em},
      literate={`\\}{{$\lambda$}}1
               {`\\\\}{{\char`\\\char`\\}}1
               {`->}{{$\rightarrow$}}2
               {`<-}{{$\leftarrow$}}2
               {`=>}{{$\Rightarrow$}}2
               {`>>}{{>>}}2
               {`>>=}{{>>=}}3
               {`>=>}{{>=>}}3
               {`|}{{$\mid$}}1]
  }

\newcommand<>{\ic}{\inlinecode}

\def\inlinecodewithtypevars{\lstinline[language=ColorHaskellWithTypeVars,
      basicstyle=\footnotesize\ttfamily,
      flexiblecolumns=false,
      basewidth={0.5em,0.45em},
      literate={`\\}{{$\lambda$}}1
               {`\\\\}{{\char`\\\char`\\}}1
               {`->}{{$\rightarrow$}}2
               {`<-}{{$\leftarrow$}}2
               {`=>}{{$\Rightarrow$}}2
               {`>>}{{>>}}2
               {`>>=}{{>>=}}3
               {`>=>}{{>=>}}3
               {`|}{{$\mid$}}1]
  }

\newcommand<>{\icwtv}{\inlinecodewithtypevars}


\newcommand<>{\type}[1]{\textcolor{hsk-types}{#1}}
\newcommand<>{\tvar}[1]{\textcolor{hsk-typevar}{#1}}
\newcommand<>{\cons}[1]{\textcolor{hsk-constructors}{#1}}
\newcommand<>{\bi}[1]{\textcolor{hsk-built-ins}{#1}}
\newcommand<>{\str}[1]{\textcolor{hsk-strings}{#1}}

\newcommand<>{\ict}[1]{\type{\ic{#1}}}
\newcommand<>{\icv}[1]{\tvar{\ic{#1}}}
\newcommand<>{\icc}[1]{\cons{\ic{#1}}}
\newcommand<>{\icbi}[1]{\bi{\ic{#1}}}



%%%%%%%%%%%%%%%%%%%%%%%%%%%%%%%%%%%%%%%%%%%%%%%%%%%%%%%%%%%%%%%%%%%%%%%%%%%%
%% Fonts

\setbeamerfont{frametitle}{size=\small}
\setbeamerfont{structure}{series=\bfseries}



%%%%%%%%%%%%%%%%%%%%%%%%%%%%%%%%%%%%%%%%%%%%%%%%%%%%%%%%%%%%%%%%%%%%%%%%%%%%
%% Settings

\setbeamertemplate{navigation symbols}{}

\bibliographystyle{apalike}

\mode<all>


\renewcommand{\(}[1]{\begin{columns}[#1]}
\renewcommand{\)}{\end{columns}}
\newcommand{\<}[1]{\begin{column}{#1}}
\renewcommand{\>}{\end{column}}

\newcommand{\Split}[3][.5]{%
\({c}%
  \<{{#1}\linewidth}%
  \begin{minipage}[c][.8\textheight]{\linewidth}%
  \begin{center}%
    {#2}%
  \end{center}%
  \end{minipage}%
  \>%
  \<{\linewidth-{#1}\linewidth}%
  \begin{minipage}[c][.8\textheight]{\linewidth}%
  \begin{center}%
    {#3}%
  \end{center}%
  \end{minipage}
  \>%
\)%
}

\newcommand<>{\Image}[3][]{%
\IfFileExists{#3.png}{%
\includegraphics#4[width=\linewidth,height=#2,keepaspectratio,#1]{#3}%
}{%
\IfFileExists{#3.jpg}{%
\includegraphics#4[width=\linewidth,height=#2,keepaspectratio,#1]{#3}%
}{%
\includegraphics#4[width=\linewidth,height=#2,keepaspectratio,#1]{img/placeholder}%
}}}

\newenvironment{TopAlign}[2][1]{\begin{minipage}[t][#2\textheight]{#1\linewidth}}{\end{minipage}}

\newcommand{\pc}[1]{{\footnotesize\texttt{#1}}}
\newcommand<>{\opc}[1]{\only#2{\pc{#1}}}
\newcommand<>{\upc}[1]{\uncover#2{\pc{#1}}}
\newcommand<>{\oic}[1]{\only#2{\ic{#1}}}
\newcommand<>{\uic}[1]{\uncover#2{\ic{#1}}}

\def\shruggie{\texttt{\raisebox{0.75em}{\char`\_}\char`\\\char`\_\kern-0.5ex(\kern-0.25ex\raisebox{0.25ex}{\rotatebox{45}{\raisebox{-.75ex}"\kern-1.5ex\rotatebox{-90})}}\kern-0.5ex)\kern-0.5ex\char`\_/\raisebox{0.75em}{\char`\_}}}


%%%%%%%%%%%%%%%%%%%%%%%%%%%%%%%%%%%%%%%%%%%%%%%%%%%%%%%%%%%%%%%%%%%%%%%%%%%%
%% Document

\begin{document}



%% Title frame

\togglefalse{showpagenumber}
\begin{frame}[fragile]
  \titlepage
\end{frame}
\toggletrue{showpagenumber}
\setcounter{framenumber}{0}



%% Intro

\section{Intro}

\begin{frame}
\frametitle{Goals}
\Split{%
  \begin{itemize}
  \item 101: basic skills
    \begin{itemize}
    \item Reading function types
    \item Pattern matching
    \item Data structures
  \end{itemize}
  \end{itemize}
  \begin{itemize}
  \item 102: first project
    \begin{itemize}
    \item Genericity
    \item IO and do notation
    \item Build a game!
  \end{itemize}
  \end{itemize}
}{%
  \Image{3cm}{img/success}
}%
\end{frame}

\begin{frame}
\frametitle{Roadmap}
\begin{TopAlign}{.52}
  \begin{itemize}
  \item 101 Recap
  \item Remaining obstacles
  \item Typeclasses overview
  \item \alt<2>{Typeclasses}{Common examples}
  \item \alt<2>{Typeclasses}{Advanced syntax}
  \item<2> Typeclasses\ldots
  \end{itemize}
\end{TopAlign}
\end{frame}

\begin{frame}
\frametitle{Checklist}
\begin{itemize}
\item Haskell 101
\item Haskell 102 codelab
\item apt-get install haskell-platform
\end{itemize}
\end{frame}



%% Recap 101

\section{Recap}

\begin{frame}
\frametitle{Curried functions, partial application}
\begin{center}
\begin{tabular}{ l c r }
  \uncover<2->{\\\ic{f}      &\ic{::}&\ic{Int -> ( Int -> [Int] )}}
  \uncover<1->{\\\ic{f}      &\ic{::}&\ic{Int ->   Int -> [Int]}}
  \uncover<3->{\\\ic{f 1}    &\ic{::}&\ic{         Int -> [Int]}}
  \uncover<5->{\\\ic{f 1 2}  &\ic{::}&\ic{                [Int]}}
  \uncover<4->{\\\ic{(f 1) 2}&\ic{::}&\ic{                [Int]}}
\end{tabular}
~\\~\\~\\
\end{center}
\end{frame}

\begin{frame}[fragile]
\frametitle{Type constructors}
\begin{center}
\begin{uncoverenv}<1->
  \begin{code}
      -- enum
      data ;\type{Bool};  = ;\cons{False}; | ;\cons{True};
      data ;\type{Color}; = ;\cons{Red}; | ;\cons{Green}; | ;\cons{Blue};
  \end{code}
\end{uncoverenv}
\begin{uncoverenv}<2->
  \begin{code}
      -- struct
      data ;\type{Point}; = ;\cons{Point}; { x :: ;\type{Double};, y :: ;\type{Double}; }
  \end{code}
\end{uncoverenv}
\begin{uncoverenv}<3->
  \begin{onlyenv}<-3>
    \begin{code}
      -- a bit more interesting
      data ;\type{Minutes}; = ;\cons{Minutes}; ;\type{Int};
      data ;\type{Maybe}; ;\tvar{a}; = ;\cons{Nothing}; | ;\cons{Just}; ;\tvar{a};
      data ;\type{List}; ;\tvar{a};  = ;\cons{Nil}; | ;\cons{Cell}; ;\tvar{a}; (;\cons{List}; ;\tvar{a};)
    \end{code}
  \end{onlyenv}
  \begin{onlyenv}<4>
    \begin{code}
      -- a bit more interesting
      data ;\type{Minutes}; = ;\cons{Minutes}; ;\type{Int};
      data ;\type{Maybe}; ;\tvar{a}; = ;\cons{Nothing}; | ;\cons{Just}; ;\tvar{a};
      data [;\tvar{a};]     = [] | (;\tvar{a};:[;\tvar{a};])
    \end{code}
  \end{onlyenv}
\end{uncoverenv}
\end{center}
\end{frame}

\begin{frame}[fragile]
\frametitle{"Deconstructors" and pattern matching}
\begin{center}
\begin{uncoverenv}<1->
  \begin{code}
      not :: ;\type{Bool}; -> ;\type{Bool};
      not ;\cons{True};  = ;\cons{False};
      not ;\cons{False}; = ;\cons{True};
  \end{code}
\end{uncoverenv}
\begin{uncoverenv}<2->
  \begin{code}
      magnitude :: ;\type{Point}; -> ;\type{Double};
      magnitude (;\cons{Point}; x y) = sqrt $ x^2 + y^2
  \end{code}%$
\end{uncoverenv}
\begin{uncoverenv}<3->
  \begin{code}
      length :: [;\tvar{a};] -> ;\type{Int};
      length []     = 0
      length (_:xs) = 1 + length xs
  \end{code}
\end{uncoverenv}
\end{center}
\end{frame}

\begin{frame}
\frametitle{We know}
\begin{center}
\begin{minipage}[c]{.8\linewidth}
\begin{itemize}
  \item how to read function types
  \item how to declare new types
  \item how to do pattern matching
\end{itemize}
\end{minipage}
\end{center}
\end{frame}



%% Shortcomings

\section{Obstacles}

\begin{frame}[fragile]
\frametitle{Our types are limited}
\begin{TopAlign}{.5}
\begin{flushleft}
               \footnotesize\texttt{\$ ghci}                                                ~\\
\uncover<1->  {\footnotesize\texttt{> \uncover<2-> {\ic{data\ }\ict{Color}\ic{\ =\ }\icc{Red}\ic{\ |\ }\icc{Green}\ic{\ |\ }\icc{Blue}}}} ~\\
\uncover<3->  {\footnotesize\texttt{> \uncover<4-> {\ic{let mycolor =\ }\icc{Red}}}}        ~\\
\begin{onlyenv}<5-7>
\uncover<5-7> {\footnotesize\texttt{> \uncover<6-7>{\ic{mycolor}}}}                         ~\\
\uncover<7>   {\footnotesize\texttt{~~~~No instance for (\ict{Show Color})}}                ~\\
\uncover<7>   {\footnotesize\texttt{~~~~arising from a use of `print'}}                     ~\\
\end{onlyenv}
\begin{onlyenv}<8-10>
\uncover<8-10>{\footnotesize\texttt{> \uncover<9-10>{\ic{mycolor ==\ }\icc{Red}}}}          ~\\
\uncover<10>  {\footnotesize\texttt{~~~~No instance for (\ict{Eq Color})}}                  ~\\
\uncover<10>  {\footnotesize\texttt{~~~~arising from a use of `(==)'}}                      ~\\
\end{onlyenv}
\end{flushleft}
\end{TopAlign}
\end{frame}

\begin{frame}[fragile]
\frametitle{Type parameters constraints?}
~\\
\begin{code}
  sumI :: [;\type{Int};] -> ;\type{Int};        sumD :: [;\type{Double};] -> ;\type{Double};
  sumI = foldl (+) 0          sumD = foldl (+) 0
\end{code}
\pause~\\
\begin{code}
                  sum :: [;\tvar{a};] -> ;\tvar{a};
                  sum = foldl (+) 0
\end{code}
\pause
\begin{flushleft}
\footnotesize
\texttt{~~No instance for (\ict{Num} \icv{a})}\\
\texttt{~~arising from a use of `(+)'}
\end{flushleft}
\end{frame}

\begin{frame}[fragile]
  \frametitle{Cascading context}
  \begin{TopAlign}{1.0}
  \begin{code}
getUser        :: ;\type{Id};   -> ;\type{Maybe}; ;\type{User};
getNextOfKin   :: ;\type{User}; -> ;\type{Maybe}; ;\type{Id};
getPhoneNumber :: ;\type{User}; -> ;\type{Maybe}; ;\type{PhoneNumber};
  \end{code}
  \begin{uncoverenv}<2->
  \begin{code}
getNextOfKinPhoneNumber :: ;\type{Id}; -> ;\type{Maybe}; ;\type{PhoneNumber};
  \end{code}
  \end{uncoverenv}
  \begin{onlyenv}<3>
  \vspace{-12pt}
  \begin{code}
getNextOfKinPhoneNumber userId =
  \end{code}
  \end{onlyenv}
  \begin{uncoverenv}<4->
  \vspace{-12pt}
  \begin{code}
getNextOfKinPhoneNumber userId = case getUser userId of
  \end{code}
  \end{uncoverenv}
  \begin{uncoverenv}<5->
  \vspace{-12pt}
  \begin{code}
    Nothing   -> Nothing
  \end{code}
  \end{uncoverenv}
  \begin{onlyenv}<6>
  \vspace{-12pt}
  \begin{code}
    Just user ->
  \end{code}
  \end{onlyenv}
  \begin{uncoverenv}<7->
  \vspace{-12pt}
  \begin{code}
    Just user -> case getNextOfKin user of
  \end{code}
  \end{uncoverenv}
  \begin{uncoverenv}<8->
  \vspace{-12pt}
  \begin{code}
        Nothing          -> Nothing
  \end{code}
  \end{uncoverenv}
  \begin{onlyenv}<9>
  \vspace{-12pt}
  \begin{code}
        Just nextOfKinId ->
  \end{code}
  \end{onlyenv}
  \begin{uncoverenv}<10->
  \vspace{-12pt}
  \begin{code}
        Just nextOfKinId -> case getUser nextOfKinId of
  \end{code}
  \end{uncoverenv}
  \begin{uncoverenv}<11->
  \vspace{-12pt}
  \begin{code}
            Nothing        -> Nothing
  \end{code}
  \end{uncoverenv}
  \begin{onlyenv}<12>
  \vspace{-12pt}
  \begin{code}
            Just nextOfKin ->
  \end{code}
  \end{onlyenv}
  \begin{uncoverenv}<13->
  \vspace{-12pt}
  \begin{code}
            Just nextOfKin -> getPhoneNumber nextOfKin
  \end{code}
  \end{uncoverenv}
  \end{TopAlign}
\end{frame}

\begin{frame}
\frametitle{IO}
\begin{TopAlign}{.1}
  \begin{minipage}[c][.3\textheight]{\linewidth}%
    \begin{center}
      \begin{itemize}
      \item<1-> Can't apply regular functions to IO values
      \item<7-> Can't get values out of IO
      \item<8-> Can't pattern match on IO
      \end{itemize}
    \end{center}
  \end{minipage}
\end{TopAlign}
  \begin{minipage}[c][.5\textheight]{\linewidth}%
    \begin{center}
      \begin{onlyenv}<2-6>
        \begin{flushleft}
          \footnotesize\texttt{\$ ghci}                                                  ~\\
          \uncover<2->{\footnotesize\texttt{> \uncover<3-> {\ic{let name = getLine --}\ict{\ IO String}}}} ~\\
          \uncover<4->{\footnotesize\texttt{> \uncover<5-> {\ic{putStr \$ "Hello " ++ name ++ "!"}}}} ~\\
          \uncover<6->{\footnotesize\texttt{~~~~Expected type: `\ict{String}'}}          ~\\
          \uncover<6->{\footnotesize\texttt{~~~~~~Actual type: `\ict{IO String}'}}       ~\\
          \uncover<6->{\footnotesize\texttt{~~~~In second argument of (++)}}             ~\\
        \end{flushleft}
      \end{onlyenv}
      \begin{center}
        \only<9>{IO's implementation\\details are hidden}
      \end{center}
    \end{center}
  \end{minipage}
\end{frame}

\begin{frame}
\frametitle{We don't know}
\begin{center}
\begin{minipage}[c]{.8\linewidth}
\begin{itemize}
  \item how to extend our data types
  \item how to express type constraints
  \item how to chain contextual functions
  \item how to use IO
\end{itemize}
\end{minipage}
\end{center}
\end{frame}

\section{Typeclasses}

\begin{frame}[fragile]
\frametitle{Typeclasses}
\begin{code}
    class ;\type{Show}; ;\tvar{a}; where
        show :: ;\tvar{a}; -> ;\type{String};
\end{code}
\begin{uncoverenv}<2->
\begin{code}
    data ;\type{Color}; = ;\cons{Red}; | ;\cons{Green}; | ;\cons{Blue};
\end{code}
\end{uncoverenv}
\begin{uncoverenv}<3->
\begin{code}
    instance ;\type{Show}; ;\type{Color}; where
        show ;\cons{Red};   = "Red"
        show ;\cons{Green}; = "Green"
        show ;\cons{Blue};  = "Blue"
\end{code}
\end{uncoverenv}
\end{frame}

\begin{frame}[fragile]
\frametitle{Constraints}
\begin{code}
  show :: ;\type{Show}; ;\tvar{a}; => ;\tvar{a}; -> ;\type{String};
\end{code}
\vspace{-12pt}
\begin{uncoverenv}<2->
\begin{code}
  sum  :: ;\type{Num}; ;\tvar{a};  => [;\tvar{a};] -> ;\tvar{a};
\end{code}
\end{uncoverenv}
\begin{uncoverenv}<3->
\vspace{-12pt}
\begin{code}
  (==) :: ;\type{Eq}; ;\tvar{a};   => ;\tvar{a}; -> ;\tvar{a}; -> ;\type{Bool};
\end{code}
\end{uncoverenv}
\begin{onlyenv}<4-6>
\begin{code}
  instance ;\type{Show}; (;\type{Maybe}; ;\tvar{a};) where
\end{code}
\end{onlyenv}
\begin{onlyenv}<7->
\begin{code}
  instance ;\type{Show}; ;\tvar{a}; => ;\type{Show}; (;\type{Maybe}; ;\tvar{a};) where
\end{code}
\end{onlyenv}
\begin{uncoverenv}<5->
\vspace{-12pt}
\begin{code}
      show ;\cons{Nothing};  = "Nothing"
\end{code}
\end{uncoverenv}
\begin{uncoverenv}<6->
\vspace{-12pt}
\begin{code}
      show (;\cons{Just}; x) = "Just " ++ show x
\end{code}
\end{uncoverenv}
\end{frame}

\begin{frame}[fragile]
\frametitle{Quick tour}
\frametitle<9->{Quick boring tour...}
\begin{center}
\begin{minipage}[t][.2\textheight]{.8\linewidth}
\begin{center}
\begin{tabular}{ c c c c c c }
\alt<2-5>{Show}{\color{lightgray}Show}&
\alt<4-5>{Read}{\color{lightgray}Read}&
\alt<6-9>{Eq}{\color{lightgray}Eq}&
\alt<10-11>{Ord}{\color{lightgray}Ord}&
\alt<12>{Bounded}{\color{lightgray}Bounded}&
\alt<13>{Enum}{\color{lightgray}Enum}
\end{tabular}
\hrule
\end{center}
\end{minipage}
~\\
\begin{minipage}[t][.6\textheight]{.8\linewidth}
  \begin{code}
    data ;\type{Color}; = ;\cons{Red}; | ;\cons{Green}; | ;\cons{Blue};
  \end{code}
  \begin{onlyenv}<2-5>
    \begin{flushleft}\footnotesize
~\\\uncover<2->{\texttt{> show \cons{Blue}}}
~\\\uncover<3->{\texttt{\str{"Blue"}}}
~\\\uncover<4->{\texttt{> read \str{"Green"} :: \type{Color}}}
~\\\uncover<5->{\texttt{\cons{Green}}}
    \end{flushleft}
  \end{onlyenv}
  \begin{onlyenv}<6>
    \begin{code}
class ;\type{Eq}; ;\tvar{a}; where
    (==) :: ;\tvar{a}; -> ;\tvar{a}; -> ;\type{Bool};

    (/=) :: ;\tvar{a}; -> ;\tvar{a}; -> ;\type{Bool};
    \end{code}
  \end{onlyenv}
  \begin{onlyenv}<7>
    \begin{code}
class ;\type{Eq}; ;\tvar{a}; where
    (==) :: ;\tvar{a}; -> ;\tvar{a}; -> ;\type{Bool};
    (==) a b = not $ a /= b
    (/=) :: ;\tvar{a}; -> ;\tvar{a}; -> ;\type{Bool};
    (/=) a b = not $ a == b
    \end{code}
  \end{onlyenv}
  \begin{onlyenv}<8>
    \begin{code}
class ;\type{Eq}; ;\tvar{a}; where
    (==) :: ;\tvar{a}; -> ;\tvar{a}; -> ;\type{Bool};
    a == b = not $ a /= b
    (/=) :: ;\tvar{a}; -> ;\tvar{a}; -> ;\type{Bool};
    a /= b = not $ a == b
    \end{code}
  \end{onlyenv}
  \begin{onlyenv}<9>
    \begin{code}
instance ;\type{Eq}; ;\type{Color}; where
    ;\cons{Red};   == ;\cons{Red};   = ;\cons{True};
    ;\cons{Green}; == ;\cons{Green}; = ;\cons{True};
    ;\cons{Blue};  == ;\cons{Blue};  = ;\cons{True};
    _     == _     = ;\cons{False};
    \end{code}
  \end{onlyenv}
  \begin{onlyenv}<10>
    \begin{code}
class ;\type{Eq}; ;\tvar{a}; => ;\type{Ord}; ;\tvar{a}; where
    compare :: ;\tvar{a}; -> ;\tvar{a}; -> ;\type{Ordering};
    (<=)    :: ;\tvar{a}; -> ;\tvar{a}; -> ;\type{Bool};
    (>=)    :: ;\tvar{a}; -> ;\tvar{a}; -> ;\type{Bool};
    (<)     :: ;\tvar{a}; -> ;\tvar{a}; -> ;\type{Bool};
    (>)     :: ;\tvar{a}; -> ;\tvar{a}; -> ;\type{Bool};
    max     :: ;\tvar{a}; -> ;\tvar{a}; -> ;\tvar{a};
    min     :: ;\tvar{a}; -> ;\tvar{a}; -> ;\tvar{a};
    \end{code}
  \end{onlyenv}
  \begin{onlyenv}<11>
    \begin{code}
class ;\type{Eq}; ;\tvar{a}; => ;\type{Ord}; ;\tvar{a}; where
    compare :: ;\tvar{a}; -> ;\tvar{a}; -> ;\type{Ordering};
    (<=)    :: ;\tvar{a}; -> ;\tvar{a}; -> ;\type{Bool};
    \end{code}
  \end{onlyenv}
  \begin{onlyenv}<12>
    \begin{code}
class ;\type{Bounded}; ;\tvar{a}; where
    minBound :: ;\tvar{a};
    maxBound :: ;\tvar{a};
    \end{code}
  \end{onlyenv}
  \begin{onlyenv}<13>
    \begin{code}
class ;\type{Enum}; ;\tvar{a}; where
    succ           :: ;\tvar{a}; -> ;\tvar{a};
    pred           :: ;\tvar{a}; -> ;\tvar{a};
    toEnum         :: ;\type{Int}; -> ;\tvar{a};
    fromEnum       :: ;\tvar{a}; -> ;\type{Int};
    enumFrom       :: ;\tvar{a}; -> [;\tvar{a};]
    enumFromThen   :: ;\tvar{a}; -> ;\tvar{a}; -> [;\tvar{a};]
    enumFromTo     :: ;\tvar{a}; -> ;\tvar{a}; -> [;\tvar{a};]
    enumFromThenTo :: ;\tvar{a}; -> ;\tvar{a}; -> ;\tvar{a}; -> [;\tvar{a};]
    \end{code}
  \end{onlyenv}
\end{minipage}
\end{center}
\end{frame}

\begin{frame}[fragile]
\frametitle{Deriving}
\begin{center}%
  \begin{code}
data ;\type{Color}; = ;\cons{Red}; | ;\cons{Green}; | ;\cons{Blue};
  \end{code}
  \begin{uncoverenv}<2>
    \vspace{-12pt}
    \begin{code}
     deriving (;\type{Show};,
               ;\type{Read};,
               ;\type{Eq};,
               ;\type{Ord};,
               ;\type{Bounded};,
               ;\type{Enum};)
    \end{code}
  \end{uncoverenv}
\end{center}
\end{frame}


\section{Codelab - Sections 1 and 2}

\begin{frame}
  \frametitle{Codelab - Sections 1 and 2}
  \begin{itemize}
  \item Codelab - Sections 1 and 2
  \item Compiler
    \begin{itemize}
    \item apt-get install haskell-platform
    \end{itemize}
  \item Source
    \begin{itemize}
    \item
      {\scriptsize\texttt{git clone \textbackslash \\
          ~~https://github.com/google/haskell-trainings.git \\
          cd haskell-trainings/haskell\_102/codelab}}
    \item
      \href{https://github.com/google/haskell-trainings/releases/download/v1.1/haskell_102_codelab.zip}
      {https://github.com/google/haskell-trainings/releases/download/v1.1/haskell\_102\_codelab.zip}
    \end{itemize}
  \end{itemize}
\end{frame}


\section{Still left...}

\begin{frame}
\frametitle{We still don't know}
\begin{center}
\begin{minipage}[c]{.8\linewidth}
\begin{itemize}
  \item \st{how to extend our data types}
  \item \st{how to express type constraints}
  \item how to chain contextual functions
  \item how to use IO
\end{itemize}
\end{minipage}
\end{center}
\end{frame}


%% Links

\section{Monads}

\begin{frame}
\frametitle{Contexts / wrappers}
\begin{minipage}[c]{\linewidth}
  \begin{center}
    \begin{tabular}{ r c r }
      \uncover<1->{A          &~~~~~$\rightarrow$~~~~~& value} \\\\
      \uncover<2->{Maybe A    &~~~~~$\rightarrow$~~~~~& optional value} \\
      \uncover<3->{List A     &~~~~~$\rightarrow$~~~~~& repeated value} \\
      \uncover<4->{IO A       &~~~~~$\rightarrow$~~~~~& impure value}   \\\\
      \uncover<5->{C~~A       &~~~~~$\rightarrow$~~~~~& "contextual" value}   \\
    \end{tabular}
  \end{center}
\end{minipage}
\end{frame}

\begin{frame}
\frametitle{Similarities}
\begin{itemize}
  \uncover<1->{\item Wrapping is trivial}
  \uncover<5->{\item Unwrapping is non-trivial / destructive / impossible}
  \uncover<9->{\item What we want: to apply functions \textbf{in the context}}
\end{itemize}
\begin{flushleft}
\begin{onlyenv}<2-4>
\uncover<2-4>{\footnotesize\texttt{~~~~\ic{wrap x = [x]}}\\}
\uncover<3-4>{\footnotesize\texttt{~~~~\ic{wrap x = Just x}}\\}
\uncover<4>{\footnotesize\texttt{~~~~\ic{wrap x = return x}}\\}
\end{onlyenv}
\begin{onlyenv}<6-8>
\uncover<6->{\footnotesize\texttt{~~~~\ic{unwrap Nothing = ???}}\\}
\uncover<7->{\footnotesize\texttt{~~~~\ic{unwrap [1,2,3] = ???}}\\}
\uncover<8->{\footnotesize\texttt{~~~~\ic{unwrap getLine =} {\color{red}NOPE}}}\\
\end{onlyenv}
\end{flushleft}
\end{frame}

\begin{frame}
\frametitle{Context functions}
\begin{minipage}[c][.3\textheight]{\linewidth}%
\begin{center}
  Three standard functions to deal with contexts
\end{center}
\end{minipage}
\begin{minipage}[t][.7\textheight]{\linewidth}%
\begin{flushleft}
\vspace{1cm}
\upc<2->{~\\~~fmap :: ~~\icwtv{(a -> b) -> c a -> c b}\\}
\upc<3->{~\\~~ap~~ :: \icwtv{c (a -> b) -> c a -> c b}\\}
\upc<4->{~\\~~bind :: \icwtv{(a -> c b) -> c a -> c b}\\}
\end{flushleft}
\end{minipage}
\end{frame}

\begin{frame}
\frametitle{fmap}
\begin{TopAlign}{.4}
\begin{itemize}
  \uncover<1->{\item fmap is like map\ldots}
  \uncover<3->{\item but generalised to all contexts}
  \uncover<13->{\item and it also has an operator version}
\end{itemize}
\end{TopAlign}
\begin{center}%
\begin{minipage}[t][.6\textheight]{.72\linewidth}%
\begin{center}%
\opc<2>{~\icwtv{map :: (a -> b) -> [a] -> [b]}}%
\opc<3>{\icwtv{fmap :: (a -> b) -> c a -> c b}}%
\begin{onlyenv}<4-13>
\begin{minipage}[t]{\linewidth}%
\begin{flushleft}
  \upc<4-> {\ic{fmap show [1, 2, 3] =} \uic<5->{["1", "2", "3"]}\\}
  \upc<6-> {\ic{fmap show (}\icc{Just}\ic{\ 42) =} \uncover<7->{\icc{Just}\ic{\ "42"}}\\}
  \upc<8-> {\ic{fmap show }\icc{Nothing} \uncover<9->{\ic{  = }\icc{Nothing}}\\}
  \upc<10->{\ic{fmap length getLine =} \upc<11->{\shruggie} \uncover<12->{\ic{--} \ict{IO Int}}\\}
\end{flushleft}
\end{minipage}
\end{onlyenv}
\begin{onlyenv}<14>
\begin{minipage}[t]{\linewidth}%
\begin{flushleft}
  \pc{\ic{show  <$> [1, 2, 3] =} \ic{["1", "2", "3"]}\\}
  \pc{\ic{show  <$> (}\icc{Just}\ic{\ 42) =} \icc{Just}\ic{\ "42"}\\}
  \pc{\ic{show  <$> }\icc{Nothing}\ic{   = }\icc{Nothing}\\}
  \pc{\ic{length <$>  getLine =} \pc{\shruggie} \ic{--} \ict{IO Int}\\}
\end{flushleft}
\end{minipage}
\end{onlyenv}
\end{center}
\end{minipage}%
\end{center}
\end{frame}


\begin{frame}
\frametitle{ap(ply)}
\begin{TopAlign}{.4}
\begin{itemize}
  \uncover<1->{\item what about functions with several arguments?}
  \uncover<5->{\item apply to the rescue!}
  \uncover<8->{\item also defined as an operator}
\end{itemize}
\end{TopAlign}
\begin{center}%
\begin{minipage}[t][.6\textheight]{.72\linewidth}%
\begin{onlyenv}<1->
\begin{flushleft}
  \upc<5->  {\icwtv{ap :: c (a -> b) -> c a -> c b}\\}
  \opc<2-9> {\\\ic{fmap (+) (Just 3) =} \uic<3->{Just (3+)}\\}
  \opc<10>  {\\\ic{(+) <$> Just 3 =} \uic<3->{Just (3+)}\\}
  \opc<4-5> {\ic{Just (3+) :: Maybe (Int -> Int)}\\}
  \opc<6-8> {\ic{ap (Just (3+)) (Just 39) =} \uic<7->{Just 42}\\}
  \opc<9-10>{\ic{Just (3+) <*> Just 39 = Just 42}\\}
  \opc<11-> {\\\ic{(+)} \uic<12->{<$> Just 3} \uic<13->{<*> Just 39} \uic<14->{= Just 42}\\}%$
\end{flushleft}
\end{onlyenv}
\end{minipage}%
\end{center}
\end{frame}

\begin{frame}[fragile]
\frametitle{bind}
\begin{TopAlign}{.32}
\begin{itemize}
  \uncover<1->{\item what about chaining functions?}
  \uncover<10->{\item bind to the rescue!}
  \uncover<13->{\item of course, also has an operator}
\end{itemize}
\end{TopAlign}
\begin{center}%
  \begin{minipage}[t][.6\textheight]{\linewidth}%
    \begin{uncoverenv}<2->
      \begin{code}
          div2 :: ;\type{Int}; -> ;\type{Maybe}; ;\type{Int};
      \end{code}
    \end{uncoverenv}
    \begin{uncoverenv}<10->
      \vspace{-12pt}
      \begin{code}
          ;\alt<13->%
          {(>>=) :: \tvar{c} \tvar{a} -> (\tvar{a} -> \tvar{c} \tvar{b}) -> \tvar{c} \tvar{b}}%
          {bind :: (\tvar{a} -> \tvar{c} \tvar{b}) -> \tvar{c} \tvar{a} -> \tvar{c} \tvar{b}};
      \end{code}
    \end{uncoverenv}
    \begin{onlyenv}<3-4>
      \begin{code}
          div2 42 = ;\cons{Just}; 21
      \end{code}
    \end{onlyenv}
    \begin{onlyenv}<4>
      \vspace{-12pt}
      \begin{code}
          div2 21 = ;\cons{Nothing};
      \end{code}
    \end{onlyenv}
    \begin{onlyenv}<5-14>
      \begin{code}
          div4 :: ;\type{Int}; -> ;\type{Maybe}; ;\type{Int};
      \end{code}
    \end{onlyenv}
    \begin{onlyenv}<6>
      \vspace{-12pt}
      \begin{code}
          div4 x =
      \end{code}
    \end{onlyenv}
    \begin{onlyenv}<7>
      \vspace{-12pt}
      \begin{code}
          div4 x = let y = div2 x
                   in
      \end{code}
    \end{onlyenv}
    \begin{onlyenv}<8>
      \vspace{-12pt}
      \begin{code}
          div4 x = let y = div2 x
                   in fmap div2 y
      \end{code}
    \end{onlyenv}
    \begin{onlyenv}<9-10>
      \vspace{-12pt}
      \begin{code}
          div4 x = let y = div2 x -- ;\type{Maybe}; ;\type{Int};
                   in fmap div2 y -- ;\type{Maybe}; (;\type{Maybe}; ;\type{Int};)
      \end{code}
    \end{onlyenv}
    \begin{onlyenv}<11>
      \vspace{-12pt}
      \begin{code}
          div4 x = let y = div2 x -- ;\type{Maybe}; ;\type{Int};
                   in bind div2 y -- ;\type{Maybe}; ;\type{Int};
      \end{code}
    \end{onlyenv}
    \begin{onlyenv}<12-13>
      \vspace{-12pt}
      \begin{code}
          div4 x = bind div2 $ div2 x
      \end{code}%$
    \end{onlyenv}
    \begin{onlyenv}<14>
      \vspace{-12pt}
      \begin{code}
          div4 x = div2 x >>= div2
      \end{code}
    \end{onlyenv}
    \begin{onlyenv}<15>
      \begin{code}
          div8 :: ;\type{Int}; -> ;\type{Maybe}; ;\type{Int};
          div8 x = div2 x >>= div2 >>= div2
      \end{code}
    \end{onlyenv}
    \begin{onlyenv}<16>
      \begin{code}
          div16 :: ;\type{Int}; -> ;\type{Maybe}; ;\type{Int};
          div16 x = div2 x >>= div2 >>= div2 >>= div2
      \end{code}
    \end{onlyenv}
  \end{minipage}%
\end{center}
\end{frame}

\begin{frame}[fragile]
  \frametitle{bind - continued}
  \begin{minipage}[t][.8\textheight]{\linewidth}
    \vspace{-0.8cm}
    \begin{code}
getUser        :: ;\type{Id};   -> ;\type{Maybe}; ;\type{User};
getNextOfKin   :: ;\type{User}; -> ;\type{Maybe}; ;\type{Id};
getPhoneNumber :: ;\type{User}; -> ;\type{Maybe}; ;\type{PhoneNumber};
    \end{code}
    \begin{uncoverenv}<3->
      \vspace{-12pt}
      \begin{code}
;\alt<4->%
{(>>=) :: \tvar{c} \tvar{a} -> (\tvar{a} -> \tvar{c} \tvar{b}) -> \tvar{c} \tvar{b}}%
{bind :: (\tvar{a} -> \tvar{c} \tvar{b}) -> \tvar{c} \tvar{a} -> \tvar{c} \tvar{b}};
      \end{code}
    \end{uncoverenv}
    \begin{uncoverenv}<2->
      \begin{code}
getNextOfKinPhoneNumber :: ;\type{Id}; -> ;\type{Maybe}; ;\type{PhoneNumber};
      \end{code}
    \end{uncoverenv}
    \begin{uncoverenv}<5->
      \vspace{-12pt}
      \begin{code}
getNextOfKinPhoneNumber uid =
      \end{code}
    \end{uncoverenv}
    \begin{uncoverenv}<6->
      \vspace{-12pt}
      \begin{code}
      getUser uid    -- ;\type{Maybe}; ;\type{User};
      \end{code}
    \end{uncoverenv}
    \begin{uncoverenv}<7->
      \vspace{-12pt}
      \begin{code}
  >>= getNextOfKin   -- ;\type{Maybe}; ;\type{Id};
      \end{code}
    \end{uncoverenv}
    \begin{uncoverenv}<8->
      \vspace{-12pt}
      \begin{code}
  >>= getUser        -- ;\type{Maybe}; ;\type{User};
      \end{code}
    \end{uncoverenv}
    \begin{uncoverenv}<9->
      \vspace{-12pt}
      \begin{code}
  >>= getPhoneNumber
      \end{code}
    \end{uncoverenv}
  \end{minipage}
\end{frame}

\def\Sep{~~~~~&~~~~~}
\begin{frame}
\frametitle{The typeclasses}
\begin{onlyenv}<1-2>
  \begin{center}
    \pc{fmap :: ~~\icwtv{(a -> b) -> c a -> c b}\\}
    \pc{ap~~ :: \icwtv{c (a -> b) -> c a -> c b}\\}
    \pc{bind :: \icwtv{(a -> c b) -> c a -> c b}\\}

    \uncover<2>{~\\But what about the typeclasses?}
  \end{center}
\end{onlyenv}
\begin{onlyenv}<3-7,9->
  \begin{center}
    ~\\
    \begin{tabular}{ c l c l c }
\uncover<3->{                  \Sep\pc{fmap}\Sep$\rightarrow$\Sep\upc<4->{Functor    }\Sep                  \\}
\uncover<5->{$^\downrcurvearrow$\Sep\pc{ap}  \Sep$\rightarrow$\Sep\upc<6->{Applicative}\Sep$^\downlcurvearrow$\\}
\uncover<7->{$^\downrcurvearrow$\Sep\pc{bind}\Sep$\rightarrow$\Sep\upc<8->{Monad}      \Sep$^\downlcurvearrow$\\}
    \end{tabular}
  \end{center}
\end{onlyenv}
\begin{onlyenv}<8>
\begin{minipage}[c][.8\textheight]{\linewidth}
\begin{center}
{\fontsize{150}{60}\selectfont MONAD}\\
\end{center}
\end{minipage}
\end{onlyenv}
\end{frame}

\begin{frame}
\frametitle{We still don't know}
\begin{center}
\begin{minipage}[c]{.8\linewidth}
\begin{itemize}
  \item \st{how to extend our data types}
  \item \st{how to express type constraints}
  \item \st{how to chain contextual functions}
  \item how to use IO
\end{itemize}
\end{minipage}
\end{center}
\end{frame}


\section{Monadic syntax}

\begin{frame}[fragile]
\frametitle{Monad}
\begin{minipage}[t][.5\textheight]{\linewidth}
\begin{center}
  \begin{codewithtvars}
    class ;\type{Applicative}; m => ;\type{Monad}; m where
        return :: a -> m a
        (>>=)  :: m a -> (a -> m b) -> m b
  \end{codewithtvars}
\end{center}
\end{minipage}
\begin{minipage}[t][.3\textheight]{\linewidth}
\begin{center}
  The basic building block
\end{center}
\end{minipage}
\end{frame}

\begin{frame}[fragile]
\frametitle{Do notation}
\begin{center}
\begin{minipage}[t][\textheight]{.8\linewidth}
\vspace{-0.8cm}
\begin{onlyenv}<1>
\begin{code}
getFirstName :: IO String
getLastName  :: IO String
greetUser    :: String -> String -> IO ()
main         :: IO ()

main =
\end{code}
\end{onlyenv}
\begin{onlyenv}<2>
\begin{code}
getFirstName :: IO String
getLastName  :: IO String
greetUser    :: String -> String -> IO ()
main         :: IO ()

main =
    getFirstName
\end{code}
\end{onlyenv}
\begin{onlyenv}<3>
\begin{code}
getFirstName :: IO String
getLastName  :: IO String
greetUser    :: String -> String -> IO ()
main         :: IO ()

main =
    getFirstName >>=
\end{code}
\end{onlyenv}
\begin{onlyenv}<4>
\begin{code}5
getFirstName :: IO String
getLastName  :: IO String
greetUser    :: String -> String -> IO ()
main         :: IO ()

main =
    getFirstName >>= \ firstname ->
\end{code}
\end{onlyenv}
\begin{onlyenv}<5>
\begin{code}
getFirstName :: IO String
getLastName  :: IO String
greetUser    :: String -> String -> IO ()
main         :: IO ()

main =
    getFirstName >>= \ firstname ->
        getLastName
\end{code}
\end{onlyenv}
\begin{onlyenv}<6>
\begin{code}
getFirstName :: IO String
getLastName  :: IO String
greetUser    :: String -> String -> IO ()
main         :: IO ()

main =
    getFirstName >>= \ firstname ->
        getLastName >>=
\end{code}
\end{onlyenv}
\begin{onlyenv}<7>
\begin{code}
getFirstName :: IO String
getLastName  :: IO String
greetUser    :: String -> String -> IO ()
main         :: IO ()

main =
    getFirstName >>= \ firstname ->
        getLastName >>= \ lastname ->
\end{code}
\end{onlyenv}
\begin{onlyenv}<8>
\begin{code}
getFirstName :: IO String
getLastName  :: IO String
greetUser    :: String -> String -> IO ()
main         :: IO ()

main =
    getFirstName >>= \ firstname ->
        getLastName >>= \ lastname ->
            greetUser firstname lastname
\end{code}
\end{onlyenv}
\begin{onlyenv}<9>
\begin{code}
getFirstName :: IO String
getLastName  :: IO String
greetUser    :: String -> String -> IO ()
main         :: IO ()

main =
    getFirstName       firstname
        getLastName       lastname
            greetUser firstname lastname
\end{code}
\end{onlyenv}
\begin{onlyenv}<10>
\begin{code}
getFirstName :: IO String
getLastName  :: IO String
greetUser    :: String -> String -> IO ()
main         :: IO ()

main =
    getFirstName ->    firstname
        getLastName ->    lastname
            greetUser firstname lastname
\end{code}
\end{onlyenv}
\begin{onlyenv}<11>
\begin{code}
getFirstName :: IO String
getLastName  :: IO String
greetUser    :: String -> String -> IO ()
main         :: IO ()

main =
    getFirstName >>= \ firstname ->
        getLastName >>= \ lastname ->
            greetUser firstname lastname

main =
\end{code}
\end{onlyenv}
\begin{onlyenv}<12>
\begin{code}
getFirstName :: IO String
getLastName  :: IO String
greetUser    :: String -> String -> IO ()
main         :: IO ()

main =
    getFirstName >>= \ firstname ->
        getLastName >>= \ lastname ->
            greetUser firstname lastname

main = do
\end{code}
\end{onlyenv}
\begin{onlyenv}<13>
\begin{code}
getFirstName :: IO String
getLastName  :: IO String
greetUser    :: String -> String -> IO ()
main         :: IO ()

main =
    getFirstName >>= \ firstname ->
        getLastName >>= \ lastname ->
            greetUser firstname lastname

main = do
    firstname <- getFirstName
\end{code}
\end{onlyenv}
\begin{onlyenv}<14>
\begin{code}
getFirstName :: IO String
getLastName  :: IO String
greetUser    :: String -> String -> IO ()
main         :: IO ()

main =
    getFirstName >>= \ firstname ->
        getLastName >>= \ lastname ->
            greetUser firstname lastname

main = do
    firstname <- getFirstName
    lastname  <- getLastName
\end{code}
\end{onlyenv}
\begin{onlyenv}<15>
\begin{code}
getFirstName :: IO String
getLastName  :: IO String
greetUser    :: String -> String -> IO ()
main         :: IO ()

main =
    getFirstName >>= \ firstname ->
        getLastName >>= \ lastname ->
            greetUser firstname lastname

main = do
    firstname <- getFirstName
    lastname  <- getLastName
    greetUser firstname lastname
\end{code}
\end{onlyenv}
\end{minipage}
\end{center}
\end{frame}

\begin{frame}
\frametitle{Recap}
\begin{itemize}
\item typeclasses to extend our data types
\item typeclasses to express type constraints
\item monadic operators to chain contextual functions
\item do notation to use IO
\end{itemize}
\end{frame}


\begin{frame}
  \frametitle{The end of the theoretical part!}
  \begin{center}
    Questions?
  \end{center}
\end{frame}


%% Links

\begin{frame}
  \frametitle{Links}
  \begin{itemize}
  \item \href{http://go/haskell-beginners-chat}{go/haskell-beginners-chat}
  \item \href{http://go/haskell-102-slides-mtv}{go/haskell-102-slides-mtv}
  \item \href{http://adit.io/posts/2013-04-17-functors,_applicatives,_and_monads_in_pictures.html}{adit.io} (functors, applicatives, and monads in pictures)
  \item \href{http://dev.stephendiehl.com/hask/}{dev.stephendiehl.com/hask/} (what I wish I knew)
  \item \href{https://www.forbes.com/sites/quora/2018/07/27/why-purely-functional-programming-is-a-great-idea-with-a-misleading-name/}{Forbes - Why Purely Functional Programming Is A Great Idea With A Misleading Name}
  \end{itemize}
\end{frame}

%% Codelab

\section{Codelab - Sections 3 and 4}

\begin{frame}
  \frametitle{Codelab - Sections 3 and 4}
  \begin{itemize}
    \item Codelab - Sections 3 and 4
  \end{itemize}
\end{frame}

\section{The end}

\begin{frame}
  \frametitle{The end!}
  \begin{center}
    Questions?
  \end{center}
\end{frame}

%%%%%%%%%%%%%%%%%%%%%%%%%%%%%%%%%%%%%%%%%%%%%%%%%%%%%%%%%%%%%%%%%%%%%%%%%%%%
%% End

\end{document}
